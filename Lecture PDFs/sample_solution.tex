\documentclass[10pt,addpoints]{exam}
\mathchardef\mhyphen="2D
\usepackage{amsfonts,amssymb,amsmath,amsthm,verbatim,enumitem}
\usepackage{graphicx}
\usepackage{systeme}
\usepackage{pgf,tikz,pgfplots}
\usepackage{algorithm,algpseudocode}
\usepackage{hyperref}
\pgfplotsset{compat=1.15}
\usepgfplotslibrary{fillbetween}
\usepackage{mathrsfs}
\usetikzlibrary{arrows}
\usetikzlibrary{calc}
\newcommand*{\rom}[1]{\expandafter\@slowromancap\romannumeral #1@}
\usepackage{enumitem}
\usepackage{tfrupee}
\newcommand{\encode}{\mathsf{Encode}}
\newcommand{\noise}{\mathsf{Noise}}
\newcommand{\decode}{\mathsf{Decode}}
\newcommand{\Z}{\mathbb{Z}}
\newcommand{\prob}[1]{\Pr\Big[ #1 \Big]}
\newcommand{\bigprob}[2]{\Pr_{#1}\left[ #2 \right]}

\newcommand{\calA}{\mathcal{A}}
\newcommand{\calB}{\mathcal{B}}
\newcommand{\calC}{\mathcal{C}}
\newcommand{\calD}{\mathcal{D}}
\newcommand{\calE}{\mathcal{E}}
\newcommand{\calK}{\mathcal{K}}
\newcommand{\calM}{\mathcal{M}}
\newcommand{\calR}{\mathcal{R}}

\newcommand{\enc}{\mathsf{Enc}}
\newcommand{\dec}{\mathsf{Dec}}
\newcommand{\negl}{\mathsf{negl}}
\newcommand{\noqss}{\mathsf{No\mhyphen Query \mhyphen Semantic \mhyphen Security}}

\newcommand{\ct}{\mathsf{ct}}

\theoremstyle{definition}
\newtheorem{theorem}{Theorem}[section]
\newtheorem{claim}[theorem]{Claim}
\newtheorem{fact}[theorem]{Fact}
\newtheorem{definition}[theorem]{Definition}
\newtheorem{corollary}[theorem]{Corollary}
\newtheorem{lemma}[theorem]{Lemma}
\newtheorem{remark}[theorem]{Remark}
\newtheorem{construction}[theorem]{Construction}
\newtheorem{assumption}[theorem]{Assumption}
\newtheorem{observation}[theorem]{Observation}

\newtheorem{innercustomclaim}{\textcolor{blue}{Claim}}
\newenvironment{customclaim}[1]
  {\renewcommand\theinnercustomclaim{\textcolor{blue}{#1}}\innercustomclaim}
  {\endinnercustomclaim}


%%%%%%%%%%%%%%%%%%%%%%%%%%%% Protocols, figures %%%%%%%%%%
\newlength{\protowidth}
\newcommand{\pprotocol}[5]{
{\begin{figure*}[#4]
\begin{center}
\setlength{\protowidth}{\textwidth}
\addtolength{\protowidth}{-3\intextsep}

\fbox{
        \small
        \hbox{\quad
        \begin{minipage}{\protowidth}
    \begin{center}
    {\bf #1}
    \end{center}
        #5
        \end{minipage}
        \quad}
        }
        \caption{\label{#3} #2}
\end{center}
\vspace{-4ex}
\end{figure*}
} }

%%%%%% comments and proof sketch
\newcommand{\jai}[1]{\textcolor{magenta}{[Jai: #1]}}
\newcommand{\vk}[1]{\textcolor{blue}{[VK: #1]}}

\newcommand{\protocol}[4]{
\pprotocol{#1}{#2}{#3}{tbh!}{#4} }





\pagestyle{head}

\firstpageheader{2201-COL759 \\ IIT Delhi}{Assignment 0}{Jai Arora \\ Venkata K}
\firstpageheadrule
\newcommand{\relatedprg}{\mathsf{Related\mhyphen PRG}}
\begin{document}

\vspace{0.4cm}

\section*{Pseudorandom Generators with Related Key Security}

\textbf{Problem Statement:}


In Lecture 05, we discussed the notion of pseudorandom generators. A length-doubling pseudorandom generator is a deterministic function $G : \{0,1\}^n \to \{0,1\}^{2n}$, and for all p.p.t. adversaries $\calA$, there exists a negligible function $\negl(\cdot)$ such that for all $n$,
\[ \Pr[\calA \text{ wins the PRG security game}] \leq 1/2 + \negl(n). \]


Recall, we discussed that PRGs may not be secure if the adversary sees the outputs on `related seeds'. In this exercise, we define a special case of PRG security w.r.t. related seeds. Let $G:\{0,1\}^n \to \{0,1\}^{\ell}$, with $\ell > n$. Consider the following security game between a challenger and an adversary: 

\protocol{$\relatedprg$}{Related Seed PRG Security Game}{relatedPRG}{
        \begin{enumerate}
            \item The challenger chooses a uniformly random bit $b \gets \{0,1\}$. 

            If $b=0$, the challenger chooses a seed $s \gets \{0,1\}^n$, sets $s'=s \oplus 0\ldots 0 1$,\footnote{The string $s'$ is same as $s$, except that the last bit is flipped.} and sends $u_1=G(s)$, $u_2 = G(s')$.

            If $b=1$, the challenger chooses two uniformly random strings $u_1, u_2 \gets \{0,1\}^{\ell}$ and sends $u_1, u_2$ to $\calA$. 

            \item The adversary sends its guess $b'$, and wins the security game if $b=b'$. 
        \end{enumerate}
    }

A length expanding function $G:\{0,1\}^n \to \{0,1\}^{\ell}$ (with $\ell>n$) is said to satisfy pseudorandomness security with related seeds if, for any prob. poly. time (p.p.t.) adversary $\calA$, there exists a negligible function $\mu(\cdot)$ such that for all $n$, $$ \prob{\calA \text{ wins in the Related Seed PRG Security Game}} \leq 1/2 + \mu(n).$$

We will show that PRG security does not imply pseudorandomness security with related seeds. Let $G : \{0,1\}^n \to \{0,1\}^{2n}$ be a secure pseudorandom generator. Construct a new length expanding function $G'$ with appropriate input/output space such that $G'$ is also a secure pseudorandom generator (assuming $G$ is a secure pseudorandom generator), but $G'$ does not satisfy pseudorandomness with related seeds.

\begin{enumerate}
    \item Construct $G'$. Your construction should use $G$ as a building block. 
    \item Show that $G'$ is a secure pseudorandom generator. That is, if there exists a p.p.t. adversary $\calA$ and a non-negligible function $\epsilon$ such that $$\Pr[\calA \text{ wins the PRG security game against }G'] = 1/2 + \epsilon,$$ then there exists a p.p.t. algorithm $\calB$ and a non-negligible function $\epsilon'$ such that $$\Pr[\calB \text{ wins the PRG security game against }G] = 1/2 + \epsilon'.$$
    
    \item Show that $G'$ does not satisfy security pseudorandomness security with related keys. 
\end{enumerate}


\vspace{10pt}

\paragraph{Note:} As mentioned in the question, you are allowed to set the input and output domains appropriately. In particular, if the security parameter is $n$, the input space can be $\{0,1\}^{p(n)}$ for any polynomial $p(\cdot)$. The $\relatedprg$ security game is defined for the case where the input domain is $\{0,1\}^n$. If the input domain was $\{0,1\}^{p(n)}$, then you would appropriately change the security game.  

\vspace{10pt}

\textbf{Solution: }
\begin{enumerate}
    \item 
    Let $G' : \{0,1\}^{2n} \to \{0,1\}^{3n}$ be defined as follows: 
    \[
        G'(s_1 ~||~ s_2) = G(s_1) ~||~ s_2
    \]
    Here, $s_1$ (resp. $s_2$) represent the first (resp. last) $n$ bits of the input, $||$ denotes string concatenation. 


    \item We will prove that $G'$ is a secure pseudorandom generator, assuming $G$ is. 

    \begin{customclaim}{1}
        \textcolor{blue}{Suppose there exists a p.p.t. adversary $\calA$ that breaks the PRG security of $G'$ with probability $1/2 + \epsilon$, where $\epsilon$ is non-negligible. Then there exists a p.p.t. algorithm $\calB$ that breaks the PRG security of $G$ with probability $1/2 + \epsilon$. }
    \end{customclaim}

    \begin{proof}
        The reduction algorithm $\calB$ is defined as follows. It receives $u \in \{0,1\}^{2n}$ from the challenger (w.r.t $G$). It then chooses a uniformly random string $s_2 \gets \{0,1\}^n$, and sends $u || s_2$ to the adversary $\calA$. The adversary sends a bit $b'$, which the reduction algorithm forwards to the challenger. 

        \vspace{5pt}

        \paragraph{Analysis of $\calB's$ success probability}

        \begin{align*}
            & \prob{\calB \text{ wins the PRG security game against }G} \\
            = & \prob{\left(\calB \text{ outputs }0\right) ~     \wedge ~ b=0} +  
                \prob{\left(\calB \text{ outputs }1\right) ~ \wedge ~ b=1} \\
            = & \prob{\left(\calA \text{ outputs }0\right) ~     \wedge ~ b=0} +  
                \prob{\left(\calA \text{ outputs }1\right) ~ \wedge ~ b=1} \\
        \end{align*}
    Now consider the following cases:
        \begin{enumerate}
            \item $b = 0$: $\calB$ receives $u = G(s) \in \{0,1\}^{2n}$ for some $s \gets \{0,1\}^n$, chooses $s_2 \gets \{0,1\}^n$ and sends $u||s_2$ to $\calA$. Note that $u||s_2 = G(s)||s_2 = G'(s||s_2)$. Now since, $s||s_2 \gets \{0,1\}^{2n}$, $\calA$ receives $u_{\calA} = G'(s')$ for some $s' \gets \{0,1\}^{2n}$.
            
            \item $b = 1$: $\calB$ receives $u \gets \{0,1\}^{2n}$, chooses $s_2 \gets \{0,1\}^n$ and sends $u||s_2$ to $\calA$, hence $u_{\calA} = u||s_2 \gets \{0,1\}^{3n}$
        \end{enumerate}
        Using these observations, we can conclude that
        \begin{align*}
            & \prob{\left(\calA \text{ outputs }0\right) ~     \wedge ~ b=0} +  
                \prob{\left(\calA \text{ outputs }1\right) ~ \wedge ~ b=1} \\
            = & \prob{\calA \text{ gets } u_{\calA} = G'(s'), ~ s' \gets \{0,1\}^{2n} ~ \wedge ~ \left(\calA \text{ outputs }0\right)} + \prob{\calA \text{ gets } u_{\calA} \gets \{0,1\}^{3n} ~ \wedge ~ \left(\calA \text{ outputs }1\right)} \\
            = & \prob{\calA \text{ wins the PRG security game against }G'} \\
            = & 1/2 + \epsilon
        \end{align*}

    \end{proof}

    \item $G'$ does not satisfy pseudorandomness security with related keys. We can construct a polynomial time adversary $\calA$ such that, given two strings $(u_1, u_2) \in \{0,1\}^{3n} \times \{0,1\}^{3n}$, $\calA$ can win the $\relatedprg$ game with probability close to $1$.  
    
    
    
    The adversary $\calA$ checks if $u_1$ and $u_2$ are identical, except for the last bit. If ${u_1} = {u_2} \oplus 0\ldots0 1$, then $\calA$ outputs $0$, else $\calA$ outputs $1$. 

    \paragraph{Analysis of $\calA's$ winning probability:} 
    \begin{align*}
        p_{\calA} = & \prob{\calA \text{ wins in the Related Seed PRG Security Game}} \\
        = &\prob{\left(\calA \text{ outputs }0\right) ~ \wedge ~ b=0} +  
        \prob{\left(\calA \text{ outputs }1\right) ~ \wedge ~ b=1}\\
        = &\frac{1}{2} + \left(\frac{1}{2} - \prob{\left(\calA \text{ outputs }0 \right) ~ \wedge ~ b=1}\right)
    \end{align*}
    
    In the last step, we use the following observation :

    $\prob{\left(\calA \text{ outputs }0 \right) ~ \wedge ~ b=1} + \prob{\left(\calA \text{ outputs }1 \right) ~ \wedge ~ b=1} = \prob{b=1} = 1/2$. 

    \vspace{5pt}

    Finally, note that if the challenger chose $b=1$ in the security experiment, then the probability that $u_1 = u_2 \oplus 0\ldots01$ is $1/2^{3n}$. Therefore, 
    $\prob{\left(\calA \text{ outputs }0 \right) ~ \wedge ~ b=1} = \frac{1}{2^{3n+1}}$.  

    \vspace{5pt}

    Therefore, $p_{\calA} = 1-1/2^{3n+1}$, and this shows that $G'$ does not satisfy $\relatedprg$ security. 

\end{enumerate}

\end{document}
